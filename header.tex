\usepackage[LGR,T2A,T1]{fontenc}
\usepackage[english]{babel}
% \usepackage[english,greek,russian]{babel}
\usepackage[utf8]{inputenc}
\usepackage[singlelinecheck=false]{caption}
\usepackage{listings}
\usepackage{shortvrb}
\usepackage{stfloats}
\usepackage{hyperref}
\usepackage{lipsum}
\usepackage[detect-all]{siunitx}

\hypersetup{
    pdfborderstyle={/S/U/W 1}, % underline links instead of boxes
    linkbordercolor=red,       % color of internal links
    citebordercolor=green,     % color of links to bibliography
    filebordercolor=magenta,   % color of file links
    urlbordercolor=brown        % color of external links
}

\captionsetup[table]{labelformat=empty,font={sf,sc,bf,},skip=0pt}
\MakeShortVerb{|}

\setcounter{tocdepth}{3}  % bookmarks up to subsubsections

\lstset{%
  basicstyle=\ttfamily,
  language=[LaTeX]{TeX},
  breaklines=true,
}

\sisetup{range-phrase = \text{ to }}

\newcommand{\version}{1.1.0}
\date{March 2023\newline{}v\version}
\author{The Geeks of RPGG}

% Define a macro for markup on the sub-table references
\newcommand\subtable[1]
{\textbf{\emph{#1}}}

% alias for the lazy typist
\newcommand\st[1]{\subtable{#1}}

\newcommand\tableheading[1]{\subsubsection{#1}}

% define commands for the table so we can mess around with the table format easily
% while this seemed like a good idea, I can't get the DndTable to work in the environment.
% \newenvironment{RandomTable}[1]
%   {
%     \subsubsection{#1}
%     \begin{DndTable}[]{c X}
%     \textbf{Roll} & \textbf{Result} \\
%   }
%   {
%     \end{DndTable}
%   }
